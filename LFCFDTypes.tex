\documentclass[12pt]{article}

%% ODER: format ==         = "\mathrel{==}"
%% ODER: format /=         = "\neq "
%
%
\makeatletter
\@ifundefined{lhs2tex.lhs2tex.sty.read}%
  {\@namedef{lhs2tex.lhs2tex.sty.read}{}%
   \newcommand\SkipToFmtEnd{}%
   \newcommand\EndFmtInput{}%
   \long\def\SkipToFmtEnd#1\EndFmtInput{}%
  }\SkipToFmtEnd

\newcommand\ReadOnlyOnce[1]{\@ifundefined{#1}{\@namedef{#1}{}}\SkipToFmtEnd}
\usepackage{amstext}
\usepackage{amssymb}
\usepackage{stmaryrd}
\DeclareFontFamily{OT1}{cmtex}{}
\DeclareFontShape{OT1}{cmtex}{m}{n}
  {<5><6><7><8>cmtex8
   <9>cmtex9
   <10><10.95><12><14.4><17.28><20.74><24.88>cmtex10}{}
\DeclareFontShape{OT1}{cmtex}{m}{it}
  {<-> ssub * cmtt/m/it}{}
\newcommand{\texfamily}{\fontfamily{cmtex}\selectfont}
\DeclareFontShape{OT1}{cmtt}{bx}{n}
  {<5><6><7><8>cmtt8
   <9>cmbtt9
   <10><10.95><12><14.4><17.28><20.74><24.88>cmbtt10}{}
\DeclareFontShape{OT1}{cmtex}{bx}{n}
  {<-> ssub * cmtt/bx/n}{}
\newcommand{\tex}[1]{\text{\texfamily#1}}	% NEU

\newcommand{\Sp}{\hskip.33334em\relax}


\newcommand{\Conid}[1]{\mathit{#1}}
\newcommand{\Varid}[1]{\mathit{#1}}
\newcommand{\anonymous}{\kern0.06em \vbox{\hrule\@width.5em}}
\newcommand{\plus}{\mathbin{+\!\!\!+}}
\newcommand{\bind}{\mathbin{>\!\!\!>\mkern-6.7mu=}}
\newcommand{\rbind}{\mathbin{=\mkern-6.7mu<\!\!\!<}}% suggested by Neil Mitchell
\newcommand{\sequ}{\mathbin{>\!\!\!>}}
\renewcommand{\leq}{\leqslant}
\renewcommand{\geq}{\geqslant}
\usepackage{polytable}

%mathindent has to be defined
\@ifundefined{mathindent}%
  {\newdimen\mathindent\mathindent\leftmargini}%
  {}%

\def\resethooks{%
  \global\let\SaveRestoreHook\empty
  \global\let\ColumnHook\empty}
\newcommand*{\savecolumns}[1][default]%
  {\g@addto@macro\SaveRestoreHook{\savecolumns[#1]}}
\newcommand*{\restorecolumns}[1][default]%
  {\g@addto@macro\SaveRestoreHook{\restorecolumns[#1]}}
\newcommand*{\aligncolumn}[2]%
  {\g@addto@macro\ColumnHook{\column{#1}{#2}}}

\resethooks

\newcommand{\onelinecommentchars}{\quad-{}- }
\newcommand{\commentbeginchars}{\enskip\{-}
\newcommand{\commentendchars}{-\}\enskip}

\newcommand{\visiblecomments}{%
  \let\onelinecomment=\onelinecommentchars
  \let\commentbegin=\commentbeginchars
  \let\commentend=\commentendchars}

\newcommand{\invisiblecomments}{%
  \let\onelinecomment=\empty
  \let\commentbegin=\empty
  \let\commentend=\empty}

\visiblecomments

\newlength{\blanklineskip}
\setlength{\blanklineskip}{0.66084ex}

\newcommand{\hsindent}[1]{\quad}% default is fixed indentation
\let\hspre\empty
\let\hspost\empty
\newcommand{\NB}{\textbf{NB}}
\newcommand{\Todo}[1]{$\langle$\textbf{To do:}~#1$\rangle$}

\EndFmtInput
\makeatother
%
%
%
%
%
%
% This package provides two environments suitable to take the place
% of hscode, called "plainhscode" and "arrayhscode". 
%
% The plain environment surrounds each code block by vertical space,
% and it uses \abovedisplayskip and \belowdisplayskip to get spacing
% similar to formulas. Note that if these dimensions are changed,
% the spacing around displayed math formulas changes as well.
% All code is indented using \leftskip.
%
% Changed 19.08.2004 to reflect changes in colorcode. Should work with
% CodeGroup.sty.
%
\ReadOnlyOnce{polycode.fmt}%
\makeatletter

\newcommand{\hsnewpar}[1]%
  {{\parskip=0pt\parindent=0pt\par\vskip #1\noindent}}

% can be used, for instance, to redefine the code size, by setting the
% command to \small or something alike
\newcommand{\hscodestyle}{}

% The command \sethscode can be used to switch the code formatting
% behaviour by mapping the hscode environment in the subst directive
% to a new LaTeX environment.

\newcommand{\sethscode}[1]%
  {\expandafter\let\expandafter\hscode\csname #1\endcsname
   \expandafter\let\expandafter\endhscode\csname end#1\endcsname}

% "compatibility" mode restores the non-polycode.fmt layout.

\newenvironment{compathscode}%
  {\par\noindent
   \advance\leftskip\mathindent
   \hscodestyle
   \let\\=\@normalcr
   \let\hspre\(\let\hspost\)%
   \pboxed}%
  {\endpboxed\)%
   \par\noindent
   \ignorespacesafterend}

\newcommand{\compaths}{\sethscode{compathscode}}

% "plain" mode is the proposed default.
% It should now work with \centering.
% This required some changes. The old version
% is still available for reference as oldplainhscode.

\newenvironment{plainhscode}%
  {\hsnewpar\abovedisplayskip
   \advance\leftskip\mathindent
   \hscodestyle
   \let\hspre\(\let\hspost\)%
   \pboxed}%
  {\endpboxed%
   \hsnewpar\belowdisplayskip
   \ignorespacesafterend}

\newenvironment{oldplainhscode}%
  {\hsnewpar\abovedisplayskip
   \advance\leftskip\mathindent
   \hscodestyle
   \let\\=\@normalcr
   \(\pboxed}%
  {\endpboxed\)%
   \hsnewpar\belowdisplayskip
   \ignorespacesafterend}

% Here, we make plainhscode the default environment.

\newcommand{\plainhs}{\sethscode{plainhscode}}
\newcommand{\oldplainhs}{\sethscode{oldplainhscode}}
\plainhs

% The arrayhscode is like plain, but makes use of polytable's
% parray environment which disallows page breaks in code blocks.

\newenvironment{arrayhscode}%
  {\hsnewpar\abovedisplayskip
   \advance\leftskip\mathindent
   \hscodestyle
   \let\\=\@normalcr
   \(\parray}%
  {\endparray\)%
   \hsnewpar\belowdisplayskip
   \ignorespacesafterend}

\newcommand{\arrayhs}{\sethscode{arrayhscode}}

% The mathhscode environment also makes use of polytable's parray 
% environment. It is supposed to be used only inside math mode 
% (I used it to typeset the type rules in my thesis).

\newenvironment{mathhscode}%
  {\parray}{\endparray}

\newcommand{\mathhs}{\sethscode{mathhscode}}

% texths is similar to mathhs, but works in text mode.

\newenvironment{texthscode}%
  {\(\parray}{\endparray\)}

\newcommand{\texths}{\sethscode{texthscode}}

% The framed environment places code in a framed box.

\def\codeframewidth{\arrayrulewidth}
\RequirePackage{calc}

\newenvironment{framedhscode}%
  {\parskip=\abovedisplayskip\par\noindent
   \hscodestyle
   \arrayrulewidth=\codeframewidth
   \tabular{@{}|p{\linewidth-2\arraycolsep-2\arrayrulewidth-2pt}|@{}}%
   \hline\framedhslinecorrect\\{-1.5ex}%
   \let\endoflinesave=\\
   \let\\=\@normalcr
   \(\pboxed}%
  {\endpboxed\)%
   \framedhslinecorrect\endoflinesave{.5ex}\hline
   \endtabular
   \parskip=\belowdisplayskip\par\noindent
   \ignorespacesafterend}

\newcommand{\framedhslinecorrect}[2]%
  {#1[#2]}

\newcommand{\framedhs}{\sethscode{framedhscode}}

% The inlinehscode environment is an experimental environment
% that can be used to typeset displayed code inline.

\newenvironment{inlinehscode}%
  {\(\def\column##1##2{}%
   \let\>\undefined\let\<\undefined\let\\\undefined
   \newcommand\>[1][]{}\newcommand\<[1][]{}\newcommand\\[1][]{}%
   \def\fromto##1##2##3{##3}%
   \def\nextline{}}{\) }%

\newcommand{\inlinehs}{\sethscode{inlinehscode}}

% The joincode environment is a separate environment that
% can be used to surround and thereby connect multiple code
% blocks.

\newenvironment{joincode}%
  {\let\orighscode=\hscode
   \let\origendhscode=\endhscode
   \def\endhscode{\def\hscode{\endgroup\def\@currenvir{hscode}\\}\begingroup}
   %\let\SaveRestoreHook=\empty
   %\let\ColumnHook=\empty
   %\let\resethooks=\empty
   \orighscode\def\hscode{\endgroup\def\@currenvir{hscode}}}%
  {\origendhscode
   \global\let\hscode=\orighscode
   \global\let\endhscode=\origendhscode}%

\makeatother
\EndFmtInput
%

\usepackage{busproofs}
\usepackage[brazil]{babel}

\title{Implementa\c c\~{a}o de Verifica\c c\~{a}o de Tipos em Haskell}

\author{Rodrigo Bonif\'{a}cio}

\begin{document}

\maketitle

\section{Introdu\c c\~{a}o}

Esse documento apresenta uma implementa\c c\~{a}o,
em \emph{literate Haskell}, do mecanismo de verifica\c c\~{a}o
de tipos de uma linguagem de programa\c c\~{a}o funcional
minimalista. Os alunos da disciplina Linguagens de Programa\c c\~{a}o
devem extender essa implementa\c c\~{a}o de tal forma que
todos os elementos sint\'{a}ticos possuam a verifica\c c\~{a}o
de tipos implementada.

\section{Vis\~{a}o geral da linguagem}


A linguagem LFCF suporta tanto express\~{o}es
identificadas (LET) quanto identificadores e fun\c c\~{o}es de alta ordem
(com o mecanismo de expressoes lambda). O foco \'{e} na
verifica\c c\~{a}o de tipos, ent\~{a}o n\~{a}o est\~{a}o
implementadas fun\c c\~{o}es voltadas para a avalia\c c\~{a}o
das express\~{o}es. 

\section{Defini\c c\~{a}o da \'{A}rvore Sint\'{a}tica Abstrata}

A implementa\c c\~{a}o consiste na defini\c c\~{a}o de
um m\'{o}dulo Haskell mais alguns tipos auxiliares, como
\texttt{Id} (um tipo sin\^{o}nimo para uma \texttt{string}) e
\texttt{Gamma}, que corresponde a um mapeamento de
identificadores em tipos. 

\begin{hscode}\SaveRestoreHook
\column{B}{@{}>{\hspre}l<{\hspost}@{}}%
\column{E}{@{}>{\hspre}l<{\hspost}@{}}%
\>[B]{}\mathbf{module}\;\Conid{LFCFDTypes}\;\mathbf{where}{}\<[E]%
\\[\blanklineskip]%
\>[B]{}\mathbf{type}\;\Conid{Id}\mathrel{=}\Conid{String}{}\<[E]%
\\[\blanklineskip]%
\>[B]{}\mathbf{type}\;\Conid{Gamma}\mathrel{=}[\mskip1.5mu (\Conid{Id},\Conid{Tipo})\mskip1.5mu]{}\<[E]%
\ColumnHook
\end{hscode}\resethooks

Os tipos v\'{a}lidos s\~{a}o definidos com o
tipo alg\'{e}brico \texttt{Tipo}, que pode
ser um tipo inteiro, um tipo booleano
e um tipo fun\c c\~{a}o. O tipo fun\c c\~{a}o
deve expressar tanto o tipo do argumento quanto
o tipo do retorno. As express\~{o}es, conforme
mencionado anteriormente, envolvem tanto
valores inteiros quanto booleanos, bem
como express\~{o}es bin\'{a}rias (soma,
subtra\c c\~{a}o, etc.), express\~{o}es
\texttt{let}, \texttt{lambda}, aplica\c c\~{a}o
de fun\c c\~{o}es e \texttt{if-then-else} 

\begin{hscode}\SaveRestoreHook
\column{B}{@{}>{\hspre}l<{\hspost}@{}}%
\column{16}{@{}>{\hspre}l<{\hspost}@{}}%
\column{E}{@{}>{\hspre}l<{\hspost}@{}}%
\>[B]{}\mathbf{data}\;\Conid{Tipo}\mathrel{=}\Conid{TInt}\mid \Conid{TBool}\mid \Conid{TFuncao}\;\Conid{Tipo}\;\Conid{Tipo}{}\<[E]%
\\
\>[B]{}\mathbf{deriving}\;(\Conid{Show},\Conid{Eq}){}\<[E]%
\\[\blanklineskip]%
\>[B]{}\mathbf{data}\;\Conid{Expressao}\mathrel{=}\Conid{ValorI}\;\Conid{Int}{}\<[E]%
\\
\>[B]{}\hsindent{16}{}\<[16]%
\>[16]{}\mid \Conid{ValorB}\;\Conid{Bool}{}\<[E]%
\\
\>[B]{}\hsindent{16}{}\<[16]%
\>[16]{}\mid \Conid{Soma}\;\Conid{Expressao}\;\Conid{Expressao}{}\<[E]%
\\
\>[B]{}\hsindent{16}{}\<[16]%
\>[16]{}\mid \Conid{Subtracao}\;\Conid{Expressao}\;\Conid{Expressao}{}\<[E]%
\\
\>[B]{}\hsindent{16}{}\<[16]%
\>[16]{}\mid \Conid{Multiplicacao}\;\Conid{Expressao}\;\Conid{Expressao}{}\<[E]%
\\
\>[B]{}\hsindent{16}{}\<[16]%
\>[16]{}\mid \Conid{Divisao}\;\Conid{Expressao}\;\Conid{Expressao}{}\<[E]%
\\
\>[B]{}\hsindent{16}{}\<[16]%
\>[16]{}\mid \Conid{Let}\;\Conid{Id}\;\Conid{Expressao}\;\Conid{Expressao}{}\<[E]%
\\
\>[B]{}\hsindent{16}{}\<[16]%
\>[16]{}\mid \Conid{Ref}\;\Conid{Id}{}\<[E]%
\\
\>[B]{}\hsindent{16}{}\<[16]%
\>[16]{}\mid \Conid{Lambda}\;(\Conid{Id},\Conid{Tipo})\;\Conid{Tipo}\;\Conid{Expressao}{}\<[E]%
\\
\>[B]{}\hsindent{16}{}\<[16]%
\>[16]{}\mid \Conid{Aplicacao}\;\Conid{Expressao}\;\Conid{Expressao}{}\<[E]%
\\
\>[B]{}\hsindent{16}{}\<[16]%
\>[16]{}\mid \Conid{If}\;\Conid{Expressao}\;\Conid{Expressao}\;\Conid{Expressao}{}\<[E]%
\\
\>[B]{}\mathbf{deriving}\;(\Conid{Show},\Conid{Eq}){}\<[E]%
\ColumnHook
\end{hscode}\resethooks

A fun\c c\~{a}o que realiza a verifica\c c\~{a}o
de tipos recebe uma express\~{a}o, um ambiente
\texttt{Gamma} e possivelmente retorna um tipo
v\'{a}lido (por isso o retorno \texttt{Maybe Tipo}).
Caso algum erro ocorra no sistema de tipos,
essa fun\c c\~{a}o deve retornar \texttt{Nothing}.
Isso permite o uso de uma nota\c c\~{a}o
baseada em monadas. 

\begin{hscode}\SaveRestoreHook
\column{B}{@{}>{\hspre}l<{\hspost}@{}}%
\column{E}{@{}>{\hspre}l<{\hspost}@{}}%
\>[B]{}\Varid{verificarTipos}\mathbin{::}\Conid{Expressao}\to \Conid{Gamma}\to \Conid{Maybe}\;\Conid{Tipo}{}\<[E]%
\ColumnHook
\end{hscode}\resethooks

Para alguns casos, a verifica\c c\~{a}o de tipos
\'{e} bem trivial, particularmente a verifica\c c\~{a}o
de tipos de express\~{o}es envolvendo valores inteiros,
valores booleanos e express\~{o}es \texttt{lambda} 

\begin{hscode}\SaveRestoreHook
\column{B}{@{}>{\hspre}l<{\hspost}@{}}%
\column{31}{@{}>{\hspre}l<{\hspost}@{}}%
\column{E}{@{}>{\hspre}l<{\hspost}@{}}%
\>[B]{}\Varid{verificarTipos}\;(\Conid{ValorI}\;\Varid{n})\;\anonymous {}\<[31]%
\>[31]{}\mathrel{=}\Varid{return}\;\Conid{TInt}{}\<[E]%
\\[\blanklineskip]%
\>[B]{}\Varid{verificarTipos}\;(\Conid{ValorB}\;\Varid{b})\;\anonymous {}\<[31]%
\>[31]{}\mathrel{=}\Varid{return}\;\Conid{TBool}{}\<[E]%
\\[\blanklineskip]%
\>[B]{}\Varid{verificarTipos}\;(\Conid{Lambda}\;(\Varid{v},\Varid{t1})\;\Varid{t2}\;\Varid{exp})\;\Varid{g}\mathrel{=}\Varid{return}\;(\Conid{TFuncao}\;\Varid{t1}\;\Varid{t2}){}\<[E]%
\ColumnHook
\end{hscode}\resethooks

Para outros casos, a verifica\c c\~{a}o de tipos
requer um certo grau de indu\c c\~{a}o (seguindo as
regras de deriva\c c\~{a}o vistas em sala de aula). Para
a soma, temos a seguinte regra de deriva\c c\~{a}o: 

\begin{prooftree}
\AxiomC{$\Gamma\vdash lhs : TInt$}
\AxiomC{$\Gamma\vdash rhs : TInt$}
\BinaryInfC{$\Gamma\vdash soma(lhs, rhs) : TInt$}
\end{prooftree}

\noindent que pode ser traduzida para Haskell como:

\begin{hscode}\SaveRestoreHook
\column{B}{@{}>{\hspre}l<{\hspost}@{}}%
\column{3}{@{}>{\hspre}l<{\hspost}@{}}%
\column{34}{@{}>{\hspre}c<{\hspost}@{}}%
\column{34E}{@{}l@{}}%
\column{E}{@{}>{\hspre}l<{\hspost}@{}}%
\>[B]{}\Varid{verificarTipos}\;(\Conid{Soma}\;\Varid{l}\;\Varid{r})\;\Varid{gamma}{}\<[34]%
\>[34]{}\mathrel{=}{}\<[34E]%
\\
\>[B]{}\hsindent{3}{}\<[3]%
\>[3]{}\Varid{verificarTipos}\;\Varid{l}\;\Varid{gamma}\bind \lambda \Varid{lt}\to {}\<[E]%
\\
\>[B]{}\hsindent{3}{}\<[3]%
\>[3]{}\Varid{verificarTipos}\;\Varid{r}\;\Varid{gamma}\bind \lambda \Varid{rt}\to {}\<[E]%
\\
\>[B]{}\hsindent{3}{}\<[3]%
\>[3]{}\mathbf{if}\;\Varid{lt}\equiv \Conid{TInt}\mathrel{\wedge}\Varid{rt}\equiv \Conid{TInt}\;\mathbf{then}\;\Varid{return}\;\Conid{TInt}\;\mathbf{else}\;\Conid{Nothing}{}\<[E]%
\ColumnHook
\end{hscode}\resethooks

Similarmente, a verifica\c c\~{a}o de express\~{o}es do tipo
\texttt{let} requer um grau de indu\c c\~{a}o. Supondo
uma express\~{a}o \texttt{let v = e in c}, primeiro verificamos
o tipo da express\~{a}o nomeda (\texttt{e}) \'{e} bem tipada com
tipo \texttt{t},
adicionamos uma associa\c c\~{a}o \texttt{(v, t)} no
ambiente \texttt{Gamma} original e computamos o tipo de
\texttt{c} no novo ambiente. Em termos de regras de deriva\c c\~{a}o,
ter\'{a}mos:

\begin{prooftree}
\AxiomC{$\Gamma\vdash e : \tau_1$}
\AxiomC{$(x,\tau_1)\Gamma\vdash c : \tau_2$}
\BinaryInfC{$\Gamma\vdash let(v,e,c) : \tau_2$}
\end{prooftree}

\noindent Em Haskell:

\begin{hscode}\SaveRestoreHook
\column{B}{@{}>{\hspre}l<{\hspost}@{}}%
\column{3}{@{}>{\hspre}l<{\hspost}@{}}%
\column{E}{@{}>{\hspre}l<{\hspost}@{}}%
\>[B]{}\Varid{verificarTipos}\;(\Conid{Let}\;\Varid{v}\;\Varid{e}\;\Varid{c})\;\Varid{gamma}\mathrel{=}{}\<[E]%
\\
\>[B]{}\hsindent{3}{}\<[3]%
\>[3]{}\Varid{verificarTipos}\;\Varid{e}\;\Varid{gamma}\bind \lambda \Varid{t}\to {}\<[E]%
\\
\>[B]{}\hsindent{3}{}\<[3]%
\>[3]{}\Varid{verificarTipos}\;\Varid{c}\;((\Varid{v},\Varid{t})\mathbin{:}\Varid{gamma}){}\<[E]%
\ColumnHook
\end{hscode}\resethooks

\section{Trabalho} 

Essa atividade do projeto final envolve verificar os tipos das demais expressoes
e escrever casos de teste para verificar se esta tudo ok. Para simplificar,
escolha a estrategia de escopo (dinamico ou estatico).



\end{document}
