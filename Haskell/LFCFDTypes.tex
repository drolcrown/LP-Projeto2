\documentclass[12pt]{article}

%% ODER: format ==         = "\mathrel{==}"
%% ODER: format /=         = "\neq "
%
%
\makeatletter
\@ifundefined{lhs2tex.lhs2tex.sty.read}%
  {\@namedef{lhs2tex.lhs2tex.sty.read}{}%
   \newcommand\SkipToFmtEnd{}%
   \newcommand\EndFmtInput{}%
   \long\def\SkipToFmtEnd#1\EndFmtInput{}%
  }\SkipToFmtEnd

\newcommand\ReadOnlyOnce[1]{\@ifundefined{#1}{\@namedef{#1}{}}\SkipToFmtEnd}
\usepackage{amstext}
\usepackage{amssymb}
\usepackage{stmaryrd}
\DeclareFontFamily{OT1}{cmtex}{}
\DeclareFontShape{OT1}{cmtex}{m}{n}
  {<5><6><7><8>cmtex8
   <9>cmtex9
   <10><10.95><12><14.4><17.28><20.74><24.88>cmtex10}{}
\DeclareFontShape{OT1}{cmtex}{m}{it}
  {<-> ssub * cmtt/m/it}{}
\newcommand{\texfamily}{\fontfamily{cmtex}\selectfont}
\DeclareFontShape{OT1}{cmtt}{bx}{n}
  {<5><6><7><8>cmtt8
   <9>cmbtt9
   <10><10.95><12><14.4><17.28><20.74><24.88>cmbtt10}{}
\DeclareFontShape{OT1}{cmtex}{bx}{n}
  {<-> ssub * cmtt/bx/n}{}
\newcommand{\tex}[1]{\text{\texfamily#1}}	% NEU

\newcommand{\Sp}{\hskip.33334em\relax}


\newcommand{\Conid}[1]{\mathit{#1}}
\newcommand{\Varid}[1]{\mathit{#1}}
\newcommand{\anonymous}{\kern0.06em \vbox{\hrule\@width.5em}}
\newcommand{\plus}{\mathbin{+\!\!\!+}}
\newcommand{\bind}{\mathbin{>\!\!\!>\mkern-6.7mu=}}
\newcommand{\rbind}{\mathbin{=\mkern-6.7mu<\!\!\!<}}% suggested by Neil Mitchell
\newcommand{\sequ}{\mathbin{>\!\!\!>}}
\renewcommand{\leq}{\leqslant}
\renewcommand{\geq}{\geqslant}
\usepackage{polytable}

%mathindent has to be defined
\@ifundefined{mathindent}%
  {\newdimen\mathindent\mathindent\leftmargini}%
  {}%

\def\resethooks{%
  \global\let\SaveRestoreHook\empty
  \global\let\ColumnHook\empty}
\newcommand*{\savecolumns}[1][default]%
  {\g@addto@macro\SaveRestoreHook{\savecolumns[#1]}}
\newcommand*{\restorecolumns}[1][default]%
  {\g@addto@macro\SaveRestoreHook{\restorecolumns[#1]}}
\newcommand*{\aligncolumn}[2]%
  {\g@addto@macro\ColumnHook{\column{#1}{#2}}}

\resethooks

\newcommand{\onelinecommentchars}{\quad-{}- }
\newcommand{\commentbeginchars}{\enskip\{-}
\newcommand{\commentendchars}{-\}\enskip}

\newcommand{\visiblecomments}{%
  \let\onelinecomment=\onelinecommentchars
  \let\commentbegin=\commentbeginchars
  \let\commentend=\commentendchars}

\newcommand{\invisiblecomments}{%
  \let\onelinecomment=\empty
  \let\commentbegin=\empty
  \let\commentend=\empty}

\visiblecomments

\newlength{\blanklineskip}
\setlength{\blanklineskip}{0.66084ex}

\newcommand{\hsindent}[1]{\quad}% default is fixed indentation
\let\hspre\empty
\let\hspost\empty
\newcommand{\NB}{\textbf{NB}}
\newcommand{\Todo}[1]{$\langle$\textbf{To do:}~#1$\rangle$}

\EndFmtInput
\makeatother
%

\usepackage[utf8]{inputenc}
\usepackage[brazil]{babel}
\usepackage{color}
\usepackage{fontenc}
\usepackage{biblatex}
\usepackage{csquotes}
\usepackage{verbatim}
\usepackage{listings}
\usepackage{busproofs}
\usepackage{amsfonts}
\usepackage[nottoc]{tocbibind}

\addbibresource{references.bib}

\title{Implementa\c c\~{a}o de Verifica\c c\~{a}o de Tipos em Haskell}

\author{Luisa Sinzker Fantin, 14/0151893\\
        Jo\~{a}o Pedro Silva Sousa, 15/0038381\\
        Rafael Oliveira de Souza, 15/0081537\\
}

\begin{document}

\maketitle

\section{Introdu\c c\~{a}o}

Esse documento apresenta uma implementa\c c\~{a}o,
em \emph{literate Haskell}, do mecanismo de verifica\c c\~{a}o
de tipos de uma linguagem de programa\c c\~{a}o funcional
minimalista.

O objetivo da implementa\c c\~{a}o de tipos consiste em identificar
erros a n\'{i}vel sint\'{a}tico (n\~{a}o erros de sintaxe) na linguagem,
utilizando o interpretador. Algumas computa\c c\~{o}es inv\'{a}lidas podem
ser identificadas antes mesmo de execut\'{a}-las, como por exemplo uma
divis\~{a}o por zero.

Caso tent\'{a}ssemos avaliar uma divis\~{a}o por zero no interpretador,
seria obtida a sa\'{i}a \emph{infinity}, resultado de uma computa\c c\~{a}o
realizada pelo interpretador da linguagem Haskell, n\~{a}o do interpretador
constru\'{i}do. Ou seja, se esper\'{a}ssemos um erro e o GHCi n\~{a}o o
identificasse, ou o GHCi marcasse um erro que n\~{a}o estamos esperando,
o interpretador n\~{a}o seria confi\'{a}vel \cite{Shriram}.


\section{Vis\~{a}o geral da linguagem}

A linguagem LFCF suporta tanto express\~{o}es
identificadas (LET) quanto identificadores e fun\c c\~{o}es de alta ordem
(com o mecanismo de expressoes lambda). O foco \'{e} na
verifica\c c\~{a}o de tipos, ent\~{a}o n\~{a}o est\~{a}o
implementadas fun\c c\~{o}es voltadas para a avalia\c c\~{a}o
das express\~{o}es. 


\section{Defini\c c\~{a}o da \'{A}rvore Sint\'{a}tica Abstrata}

A implementa\c c\~{a}o consiste na defini\c c\~{a}o de
um m\'{o}dulo Haskell mais alguns tipos auxiliares, como
\texttt{Id} (um tipo sin\^{o}nimo para uma \texttt{string}) e
\texttt{Gamma}, que corresponde a um mapeamento de
identificadores em tipos. 

\begingroup\par\noindent\advance\leftskip\mathindent\(
\begin{pboxed}\SaveRestoreHook
\column{B}{@{}>{\hspre}l<{\hspost}@{}}%
\column{E}{@{}>{\hspre}l<{\hspost}@{}}%
\>[B]{}\mathbf{module}\;\Conid{LFCFDTypes}\;\mathbf{where}{}\<[E]%
\\[\blanklineskip]%
\>[B]{}\mathbf{type}\;\Conid{Id}\mathrel{=}\Conid{String}{}\<[E]%
\\[\blanklineskip]%
\>[B]{}\mathbf{type}\;\Conid{Gamma}\mathrel{=}[\mskip1.5mu (\Conid{Id},\Conid{Tipo})\mskip1.5mu]{}\<[E]%
\ColumnHook
\end{pboxed}
\)\par\noindent\endgroup\resethooks

Os tipos v\'{a}lidos s\~{a}o definidos com o
tipo alg\'{e}brico \texttt{Tipo}, que pode
ser um tipo inteiro, um tipo booleano
e um tipo fun\c c\~{a}o. O tipo fun\c c\~{a}o
deve expressar tanto o tipo do argumento quanto
o tipo do retorno. As express\~{o}es, conforme
mencionado anteriormente, envolvem tanto
valores inteiros quanto booleanos, bem
como express\~{o}es bin\'{a}rias (soma,
subtra\c c\~{a}o, etc.), express\~{o}es
\texttt{let}, \texttt{lambda}, aplica\c c\~{a}o
de fun\c c\~{o}es e \texttt{if-then-else} 

\begingroup\par\noindent\advance\leftskip\mathindent\(
\begin{pboxed}\SaveRestoreHook
\column{B}{@{}>{\hspre}l<{\hspost}@{}}%
\column{5}{@{}>{\hspre}l<{\hspost}@{}}%
\column{11}{@{}>{\hspre}l<{\hspost}@{}}%
\column{16}{@{}>{\hspre}l<{\hspost}@{}}%
\column{E}{@{}>{\hspre}l<{\hspost}@{}}%
\>[B]{}\mathbf{data}\;\Conid{Tipo}\mathrel{=}\Conid{TInt}{}\<[E]%
\\
\>[B]{}\hsindent{11}{}\<[11]%
\>[11]{}\mid \Conid{TBool}{}\<[E]%
\\
\>[B]{}\hsindent{11}{}\<[11]%
\>[11]{}\mid \Conid{TFuncao}\;\Conid{Tipo}\;\Conid{Tipo}{}\<[E]%
\\
\>[B]{}\hsindent{5}{}\<[5]%
\>[5]{}\mathbf{deriving}\;(\Conid{Show},\Conid{Eq}){}\<[E]%
\\[\blanklineskip]%
\>[B]{}\mathbf{data}\;\Conid{Expressao}\mathrel{=}\Conid{ValorI}\;\Conid{Int}{}\<[E]%
\\
\>[B]{}\hsindent{16}{}\<[16]%
\>[16]{}\mid \Conid{ValorB}\;\Conid{Bool}{}\<[E]%
\\
\>[B]{}\hsindent{16}{}\<[16]%
\>[16]{}\mid \Conid{Soma}\;\Conid{Expressao}\;\Conid{Expressao}{}\<[E]%
\\
\>[B]{}\hsindent{16}{}\<[16]%
\>[16]{}\mid \Conid{Subtracao}\;\Conid{Expressao}\;\Conid{Expressao}{}\<[E]%
\\
\>[B]{}\hsindent{16}{}\<[16]%
\>[16]{}\mid \Conid{Multiplicacao}\;\Conid{Expressao}\;\Conid{Expressao}{}\<[E]%
\\
\>[B]{}\hsindent{16}{}\<[16]%
\>[16]{}\mid \Conid{Divisao}\;\Conid{Expressao}\;\Conid{Expressao}{}\<[E]%
\\
\>[B]{}\hsindent{16}{}\<[16]%
\>[16]{}\mid \Conid{Let}\;\Conid{Id}\;\Conid{Expressao}\;\Conid{Expressao}{}\<[E]%
\\
\>[B]{}\hsindent{16}{}\<[16]%
\>[16]{}\mid \Conid{Ref}\;\Conid{Id}{}\<[E]%
\\
\>[B]{}\hsindent{16}{}\<[16]%
\>[16]{}\mid \Conid{Lambda}\;(\Conid{Id},\Conid{Tipo})\;\Conid{Tipo}\;\Conid{Expressao}{}\<[E]%
\\
\>[B]{}\hsindent{16}{}\<[16]%
\>[16]{}\mid \Conid{Aplicacao}\;\Conid{Expressao}\;\Conid{Expressao}{}\<[E]%
\\
\>[B]{}\hsindent{16}{}\<[16]%
\>[16]{}\mid \Conid{If}\;\Conid{Expressao}\;\Conid{Expressao}\;\Conid{Expressao}{}\<[E]%
\\
\>[B]{}\hsindent{5}{}\<[5]%
\>[5]{}\mathbf{deriving}\;(\Conid{Show},\Conid{Eq}){}\<[E]%
\ColumnHook
\end{pboxed}
\)\par\noindent\endgroup\resethooks

A fun\c c\~{a}o que realiza a verifica\c c\~{a}o
de tipos recebe uma express\~{a}o, um ambiente
\texttt{Gamma} e possivelmente retorna um tipo
v\'{a}lido (por isso o retorno \texttt{Maybe Tipo}).
Caso o tipo verificado pelo interpretador seja um tipo
\texttt{a} v\'{a}lido, \'{e} retornado um \texttt{Just a},
caso algum erro ocorra no sistema de tipos,
essa fun\c c\~{a}o deve retornar \texttt{Nothing}.
Isso permite o uso de uma nota\c c\~{a}o
baseada em monadas. 

\begingroup\par\noindent\advance\leftskip\mathindent\(
\begin{pboxed}\SaveRestoreHook
\column{B}{@{}>{\hspre}l<{\hspost}@{}}%
\column{E}{@{}>{\hspre}l<{\hspost}@{}}%
\>[B]{}\Varid{verificarTipos}\mathbin{::}\Conid{Expressao}\to \Conid{Gamma}\to \Conid{Maybe}\;\Conid{Tipo}{}\<[E]%
\ColumnHook
\end{pboxed}
\)\par\noindent\endgroup\resethooks

Para alguns casos, a verifica\c c\~{a}o de tipos
\'{e} bem trivial, particularmente a verifica\c c\~{a}o
de tipos de express\~{o}es envolvendo valores inteiros,
valores booleanos e express\~{o}es \texttt{lambda}.

\begingroup\par\noindent\advance\leftskip\mathindent\(
\begin{pboxed}\SaveRestoreHook
\column{B}{@{}>{\hspre}l<{\hspost}@{}}%
\column{E}{@{}>{\hspre}l<{\hspost}@{}}%
\>[B]{}\Varid{verificarTipos}\;(\Conid{ValorI}\;\Varid{n})\;\anonymous \mathrel{=}\Varid{return}\;\Conid{TInt}{}\<[E]%
\\[\blanklineskip]%
\>[B]{}\Varid{verificarTipos}\;(\Conid{ValorB}\;\Varid{b})\;\anonymous \mathrel{=}\Varid{return}\;\Conid{TBool}{}\<[E]%
\\[\blanklineskip]%
\>[B]{}\Varid{verificarTipos}\;(\Conid{Lambda}\;(\Varid{v},\Varid{t1})\;\Varid{t2}\;\Varid{exp})\;\Varid{g}\mathrel{=}\Varid{return}\;(\Conid{TFuncao}\;\Varid{t1}\;\Varid{t2}){}\<[E]%
\ColumnHook
\end{pboxed}
\)\par\noindent\endgroup\resethooks

Os casos mostrados s\~{a}o a base dos tipos da linguagem, pois
todas as constru\c c\~{o}es devem ser reduzidas a um dos casos triviais.

Para outros casos, a verifica\c c\~{a}o de tipos
requer um certo grau de indu\c c\~{a}o (seguindo as
regras de deriva\c c\~{a}o vistas em sala de aula). Para
a soma, temos a seguinte regra de deriva\c c\~{a}o: 

\begin{prooftree}
    \AxiomC{$\Gamma\vdash lhs : \texttt{TInt}$}
    \AxiomC{$\Gamma\vdash rhs : \texttt{TInt}$}
    \BinaryInfC{$\Gamma\vdash soma(lhs, rhs) : \texttt{TInt}$}
\end{prooftree}

que pode ser traduzida para Haskell como:

\begingroup\par\noindent\advance\leftskip\mathindent\(
\begin{pboxed}\SaveRestoreHook
\column{B}{@{}>{\hspre}l<{\hspost}@{}}%
\column{5}{@{}>{\hspre}l<{\hspost}@{}}%
\column{9}{@{}>{\hspre}l<{\hspost}@{}}%
\column{34}{@{}>{\hspre}c<{\hspost}@{}}%
\column{34E}{@{}l@{}}%
\column{E}{@{}>{\hspre}l<{\hspost}@{}}%
\>[B]{}\Varid{verificarTipos}\;(\Conid{Soma}\;\Varid{l}\;\Varid{r})\;\Varid{gamma}{}\<[34]%
\>[34]{}\mathrel{=}{}\<[34E]%
\\
\>[B]{}\hsindent{5}{}\<[5]%
\>[5]{}\Varid{verificarTipos}\;\Varid{l}\;\Varid{gamma}\bind \lambda \Varid{lt}\to {}\<[E]%
\\
\>[B]{}\hsindent{5}{}\<[5]%
\>[5]{}\Varid{verificarTipos}\;\Varid{r}\;\Varid{gamma}\bind \lambda \Varid{rt}\to {}\<[E]%
\\
\>[B]{}\hsindent{5}{}\<[5]%
\>[5]{}\mathbf{if}\;\Varid{lt}\equiv \Conid{TInt}\mathrel{\wedge}\Varid{rt}\equiv \Conid{TInt}{}\<[E]%
\\
\>[5]{}\hsindent{4}{}\<[9]%
\>[9]{}\mathbf{then}\;\Varid{return}\;\Conid{TInt}{}\<[E]%
\\
\>[5]{}\hsindent{4}{}\<[9]%
\>[9]{}\mathbf{else}\;\Conid{Nothing}{}\<[E]%
\ColumnHook
\end{pboxed}
\)\par\noindent\endgroup\resethooks

Analogamente, as opera\c c\~{o}es de subtra\c c\~{a}o, multiplica\c c\~{a}o
e divis\~{a}o possuem \'{a}rvores de deriva\c c\~{a}o parecidas:

\begin{itemize}

    \item   Subtra\c c\~{a}o:
        \begin{prooftree}
            \AxiomC{$\Gamma\vdash lhs : \texttt{TInt}$}
            \AxiomC{$\Gamma\vdash rhs : \texttt{TInt}$}
            \BinaryInfC{$\Gamma\vdash subtracao(lhs, rhs) : \texttt{TInt}$}
        \end{prooftree}

    \item   Multiplica\c c\~{a}o:
        \begin{prooftree}
            \AxiomC{$\Gamma\vdash lhs : \texttt{TInt}$}
            \AxiomC{$\Gamma\vdash rhs : \texttt{TInt}$}
            \BinaryInfC{$\Gamma\vdash multiplicacao(lhs, rhs) : \texttt{TInt}$}
        \end{prooftree}

\end{itemize}

Apenas uma mudan\c ca na verifica\c c\~{a}o de tipos da divis\~{a}o:
caso seja identificado um denominador nulo (igual a zero) na divis\~{a}o,
retorna-se \texttt{Nothing}, caso contr\'{a}rio, avalia-se o tipo das
express\~{o}es alg\'{e}bricas normalmente.

\begin{prooftree}
    \AxiomC{$\Gamma\vdash lhs : \texttt{TInt}$}
    \AxiomC{$\Gamma\vdash rhs : \texttt{TInt}$}
    \AxiomC{$rhs \rightarrow \neg (ValorI 0)$}
    \TrinaryInfC{$\Gamma\vdash divisao(lhs, rhs) : \texttt{TInt}$}
\end{prooftree}

Em Haskell:

\begingroup\par\noindent\advance\leftskip\mathindent\(
\begin{pboxed}\SaveRestoreHook
\column{B}{@{}>{\hspre}l<{\hspost}@{}}%
\column{5}{@{}>{\hspre}l<{\hspost}@{}}%
\column{9}{@{}>{\hspre}l<{\hspost}@{}}%
\column{14}{@{}>{\hspre}l<{\hspost}@{}}%
\column{17}{@{}>{\hspre}l<{\hspost}@{}}%
\column{21}{@{}>{\hspre}l<{\hspost}@{}}%
\column{E}{@{}>{\hspre}l<{\hspost}@{}}%
\>[B]{}\Varid{verificarTipos}\;(\Conid{Divisao}\;\Varid{l}\;\Varid{r})\;\Varid{gamma}\mathrel{=}{}\<[E]%
\\
\>[B]{}\hsindent{5}{}\<[5]%
\>[5]{}\mathbf{if}\;\Varid{r}\equiv (\Conid{ValorI}\;\mathrm{0}){}\<[E]%
\\
\>[5]{}\hsindent{4}{}\<[9]%
\>[9]{}\mathbf{then}\;\Conid{Nothing}{}\<[E]%
\\
\>[5]{}\hsindent{4}{}\<[9]%
\>[9]{}\mathbf{else}\;\Varid{verificarTipos}\;\Varid{l}\;\Varid{gamma}\bind \lambda \Varid{lt}\to {}\<[E]%
\\
\>[9]{}\hsindent{5}{}\<[14]%
\>[14]{}\Varid{verificarTipos}\;\Varid{r}\;\Varid{gamma}\bind \lambda \Varid{rt}\to {}\<[E]%
\\
\>[14]{}\hsindent{3}{}\<[17]%
\>[17]{}\mathbf{if}\;\Varid{lt}\equiv \Conid{TInt}\mathrel{\wedge}\Varid{rt}\equiv \Conid{TInt}{}\<[E]%
\\
\>[17]{}\hsindent{4}{}\<[21]%
\>[21]{}\mathbf{then}\;\Varid{return}\;\Conid{TInt}{}\<[E]%
\\
\>[17]{}\hsindent{4}{}\<[21]%
\>[21]{}\mathbf{else}\;\Conid{Nothing}{}\<[E]%
\ColumnHook
\end{pboxed}
\)\par\noindent\endgroup\resethooks

Similarmente, a verifica\c c\~{a}o de express\~{o}es do tipo
\texttt{let} requer um grau de indu\c c\~{a}o. Supondo
uma express\~{a}o \texttt{let v = e in c}, primeiro verificamos
o tipo da express\~{a}o nomeada (\texttt{e}) \'{e} bem tipada com
tipo \texttt{t},
adicionamos uma associa\c c\~{a}o \texttt{(v, t)} no
ambiente \texttt{Gamma} original e computamos o tipo de
\texttt{c} no novo ambiente. Em termos de regras de deriva\c c\~{a}o,
teremos:

\begin{prooftree}
\AxiomC{$\Gamma\vdash e : \tau_1$}
\AxiomC{$(x,\tau_1)\Gamma\vdash c : \tau_2$}
\BinaryInfC{$\Gamma\vdash let(v,e,c) : \tau_2$}
\end{prooftree}

Em Haskell:

\begingroup\par\noindent\advance\leftskip\mathindent\(
\begin{pboxed}\SaveRestoreHook
\column{B}{@{}>{\hspre}l<{\hspost}@{}}%
\column{5}{@{}>{\hspre}l<{\hspost}@{}}%
\column{9}{@{}>{\hspre}l<{\hspost}@{}}%
\column{13}{@{}>{\hspre}l<{\hspost}@{}}%
\column{29}{@{}>{\hspre}l<{\hspost}@{}}%
\column{E}{@{}>{\hspre}l<{\hspost}@{}}%
\>[B]{}\Varid{verificarTipos}\;(\Conid{Let}\;\Varid{v}\;\Varid{e}\;\Varid{c})\;\Varid{gamma}\mathrel{=}{}\<[E]%
\\
\>[B]{}\hsindent{5}{}\<[5]%
\>[5]{}\Varid{verificarTipos}\;\Varid{e}\;\Varid{gamma}{}\<[29]%
\>[29]{}\bind \lambda \Varid{lt}\to {}\<[E]%
\\
\>[B]{}\hsindent{5}{}\<[5]%
\>[5]{}\Varid{verificarTipos}\;\Varid{c}\;\Varid{gamma'}\bind \lambda \Varid{rt}\to {}\<[E]%
\\
\>[5]{}\hsindent{4}{}\<[9]%
\>[9]{}\mathbf{if}\;\Varid{lt}\equiv \Varid{rt}{}\<[E]%
\\
\>[9]{}\hsindent{4}{}\<[13]%
\>[13]{}\mathbf{then}\;\Varid{return}\;\Varid{rt}{}\<[E]%
\\
\>[9]{}\hsindent{4}{}\<[13]%
\>[13]{}\mathbf{else}\;\Conid{Nothing}{}\<[E]%
\\
\>[B]{}\hsindent{5}{}\<[5]%
\>[5]{}\mathbf{where}{}\<[E]%
\\
\>[5]{}\hsindent{4}{}\<[9]%
\>[9]{}\Varid{gamma'}\mathrel{=}\Varid{incrementaAmb}\;\Varid{v}\;(\Varid{verificarTipos}\;\Varid{e}\;\Varid{gamma})\;\Varid{gamma}{}\<[E]%
\ColumnHook
\end{pboxed}
\)\par\noindent\endgroup\resethooks

A fun\c c\~{a}o \texttt{incrementaAmb} enriquece o ambiente na mudan\c ca
de escopo das express\~{o}es \texttt{Let}.


A verifica\c c\~{a}o do tipo de uma refer\^{e}ncia \'{e} bastante
simples, requer apenas pesquisar o identificador no ambiente de
mapeamento $\Gamma$, para isso, utilizamos a fun\c c\~{a}o pesquisar:

\begingroup\par\noindent\advance\leftskip\mathindent\(
\begin{pboxed}\SaveRestoreHook
\column{B}{@{}>{\hspre}l<{\hspost}@{}}%
\column{E}{@{}>{\hspre}l<{\hspost}@{}}%
\>[B]{}\Varid{verificarTipos}\;(\Conid{Ref}\;\Varid{v})\;\Varid{gamma}\mathrel{=}\Varid{pesquisar}\;\Varid{v}\;\Varid{gamma}{}\<[E]%
\ColumnHook
\end{pboxed}
\)\par\noindent\endgroup\resethooks

Caso o identificador n\~{a}o seja encontrado no ambiente de
mapeamento, \'{e} retornado um erro.

Uma express\~{a}o \texttt{If Expressao Expressao Expressao} \'{e}
formada por tr\^{e}s partes: a condi\c c\~{a}o de avalia\c c\~{a}o;
uma cl\'{a}usula \texttt{then} e uma cl\'{a}usula \texttt{else}.
A cl\'{a}usula \texttt{then} \'{e} executada se a express\~{a}o de
condi\c c\~{a}o for avaliada verdadeira e, a cl\'{a}usula \texttt{else},
caso contr\'{a}rio.
A determina\c c\~{a}o do tipo de uma express\~{a}o \texttt{If} \'{e}
um processo em duas etapas:

\begin{enumerate}
    \item Verificar se o tipo da express\~{a}o de condi\c c\~{a}o \'{e}
    um booleano (\texttt{TBool});
    \item Verificar se os tipos das express\~{o}es associadas \`{a}s
    cl\'{a}usulas \texttt{then} e \texttt{else} s\~{a}o iguais.
\end{enumerate}

O interpretador ser\'{a} como apresentado:

\begingroup\par\noindent\advance\leftskip\mathindent\(
\begin{pboxed}\SaveRestoreHook
\column{B}{@{}>{\hspre}l<{\hspost}@{}}%
\column{5}{@{}>{\hspre}l<{\hspost}@{}}%
\column{9}{@{}>{\hspre}l<{\hspost}@{}}%
\column{14}{@{}>{\hspre}l<{\hspost}@{}}%
\column{17}{@{}>{\hspre}l<{\hspost}@{}}%
\column{E}{@{}>{\hspre}l<{\hspost}@{}}%
\>[B]{}\Varid{verificarTipos}\;(\Conid{If}\;\Varid{test}\;\Varid{pass}\;\Varid{fail})\;\Varid{gamma}\mathrel{=}{}\<[E]%
\\
\>[B]{}\hsindent{5}{}\<[5]%
\>[5]{}\Varid{verificarTipos}\;\Varid{test}\;\Varid{gamma}\bind \lambda \Varid{t}\to {}\<[E]%
\\
\>[B]{}\hsindent{5}{}\<[5]%
\>[5]{}\mathbf{if}\;\Varid{t}\equiv \Conid{TBool}{}\<[E]%
\\
\>[5]{}\hsindent{4}{}\<[9]%
\>[9]{}\mathbf{then}\;\Varid{verificarTipos}\;\Varid{pass}\;\Varid{gamma}\bind \lambda \Varid{p}\to {}\<[E]%
\\
\>[9]{}\hsindent{5}{}\<[14]%
\>[14]{}\Varid{verificarTipos}\;\Varid{fail}\;\Varid{gamma}\bind \lambda \Varid{f}\to {}\<[E]%
\\
\>[9]{}\hsindent{5}{}\<[14]%
\>[14]{}\mathbf{if}\;\Varid{p}\equiv \Varid{f}{}\<[E]%
\\
\>[14]{}\hsindent{3}{}\<[17]%
\>[17]{}\mathbf{then}\;\Varid{return}\;\Varid{p}{}\<[E]%
\\
\>[14]{}\hsindent{3}{}\<[17]%
\>[17]{}\mathbf{else}\;\Conid{Nothing}{}\<[E]%
\\
\>[5]{}\hsindent{4}{}\<[9]%
\>[9]{}\mathbf{else}\;\Conid{Nothing}{}\<[E]%
\ColumnHook
\end{pboxed}
\)\par\noindent\endgroup\resethooks

Conclui-se ent\~{a}o que, a regra para definir uma express\~{a}o
\texttt{If} \'{e}:

\begin{prooftree}
    \AxiomC{$\Gamma\vdash cond : \texttt{boolean}$}
    \AxiomC{$\Gamma\vdash then : \tau$}
    \AxiomC{$\Gamma\vdash else : \tau$}
    \TrinaryInfC{$\Gamma\vdash \{\texttt{if}$ $cond$ $then$ $else\} : \tau$}
\end{prooftree}

A necessidade das cl\'{a}usulas \texttt{then} e \texttt{else} possuirem
o mesmo tipo (no caso da linguagem pode ser inteiro ou booleano) permite
que possamos atribuir um tipo espec\'{i}fico para express\~{o}es \texttt{If}
e que n\~{a}o dependem de uma avalia\c c\~{a}o em tempo de execu\c c\~{a}o.


Para definir o tipo de uma aplica\c c\~{a}o de fun\c c\~{a}o, ser\~{a}o
necess\'{a}rias mais verifica\c c\~{o}es. Uma aplica\c c\~{a}o \'{e}
definida como \texttt{Aplicacao Expressao Expressao}, na qual a primeira
express\~{a}o \'{e} a defini\c c\~{a}o da fun\c c\~{a}o e a segunda
representa o argumento. O processo de verifica\c c\~{a} para uma
\texttt{Aplicacao def arg} \'{e} o seguinte:

\begin{enumerate}

    \item Verificar se a express\~{a}o que define a fun\c c\~{a}o
    (\texttt{def}) \'{e} uma express\~{a}o \textit{lambda}; caso seja,
    ela automaticamente possui o tipo \texttt{TFuncao tId tExp};
    \item Verificar o tipo do argumento passado para a aplica\c c\~{a}o
    (\texttt{arg}) no contexto $\Gamma$ (denominado \texttt{tArg});
    \item Comparar \texttt{tArg} com \texttt{tId}:
    \begin{itemize}
        \item Se forem iguais, ent\~{a}o o par\^{a}metro passado na
        aplica\c c\~{a}o pode ser associado ao identificador da
        express\~{a}o \textit{lambda};
        \item Se forem diferentes, ent\~{a}o o tipo do par\^{a}metro
        difere do tipo do argumento e pode ser retornado \texttt{Nothing};
    \end{itemize}
    \item O tipo da aplica\c c\~{a}o ser\'{a} o tipo da express\~{a}o
    associada na express\~{a}o \textit{lambda} verificado no ambiente
    $\Gamma$' composto pela tupla \texttt{(id, tipo\_do\_id)};

\end{enumerate}

Em Haskell:

\begingroup\par\noindent\advance\leftskip\mathindent\(
\begin{pboxed}\SaveRestoreHook
\column{B}{@{}>{\hspre}l<{\hspost}@{}}%
\column{5}{@{}>{\hspre}l<{\hspost}@{}}%
\column{9}{@{}>{\hspre}l<{\hspost}@{}}%
\column{13}{@{}>{\hspre}l<{\hspost}@{}}%
\column{17}{@{}>{\hspre}l<{\hspost}@{}}%
\column{21}{@{}>{\hspre}l<{\hspost}@{}}%
\column{E}{@{}>{\hspre}l<{\hspost}@{}}%
\>[B]{}\Varid{verificarTipos}\;(\Conid{Aplicacao}\;\Varid{def}\;\Varid{arg})\;\Varid{gamma}\mathrel{=}{}\<[E]%
\\
\>[B]{}\hsindent{5}{}\<[5]%
\>[5]{}\mathbf{case}\;\Varid{def}\;\mathbf{of}{}\<[E]%
\\
\>[5]{}\hsindent{4}{}\<[9]%
\>[9]{}(\Conid{Lambda}\;(\Varid{v},\Varid{tId})\;\Varid{tExp}\;\Varid{exp})\to {}\<[E]%
\\
\>[9]{}\hsindent{4}{}\<[13]%
\>[13]{}\Varid{verificarTipos}\;\Varid{arg}\;\Varid{gamma}\bind \lambda \Varid{a}\to {}\<[E]%
\\
\>[13]{}\hsindent{4}{}\<[17]%
\>[17]{}\mathbf{if}\;\Varid{a}\equiv \Varid{tId}{}\<[E]%
\\
\>[17]{}\hsindent{4}{}\<[21]%
\>[21]{}\mathbf{then}\;\Varid{verificarTipos}\;\Varid{exp}\;\Varid{gamma'}{}\<[E]%
\\
\>[17]{}\hsindent{4}{}\<[21]%
\>[21]{}\mathbf{else}\;\Conid{Nothing}{}\<[E]%
\\
\>[13]{}\hsindent{4}{}\<[17]%
\>[17]{}\mathbf{where}{}\<[E]%
\\
\>[17]{}\hsindent{4}{}\<[21]%
\>[21]{}\Varid{gamma'}\mathrel{=}[\mskip1.5mu (\Varid{v},\Varid{tId})\mskip1.5mu]{}\<[E]%
\\
\>[5]{}\hsindent{4}{}\<[9]%
\>[9]{}\Varid{otherwise}\to \Varid{error}\;(\text{\tt \char34 Aplicacao~de~funcao~nao~anonima\char34}){}\<[E]%
\ColumnHook
\end{pboxed}
\)\par\noindent\endgroup\resethooks

Conclui-se ent\~{a}o que a regra que define uma aplica\c c\~{a}o de
fun\c c\~{a}o pode ser representada pela \'{a}rvore a seguir:

\begin{prooftree}
    \AxiomC{$\Gamma\vdash def : (\texttt{TFuncao}$ $\tau_{1}$ $\tau_{2})$}
    \AxiomC{$\Gamma\vdash arg : \tau_{1}$}
    \BinaryInfC{$\Gamma\vdash \{$\texttt{Aplicacao }$def$ $arg\} : \tau_{2}$}
\end{prooftree}

\begingroup\par\noindent\advance\leftskip\mathindent\(
\begin{pboxed}\SaveRestoreHook
\column{B}{@{}>{\hspre}l<{\hspost}@{}}%
\column{5}{@{}>{\hspre}l<{\hspost}@{}}%
\column{26}{@{}>{\hspre}l<{\hspost}@{}}%
\column{E}{@{}>{\hspre}l<{\hspost}@{}}%
\>[B]{}\Varid{incrementaAmb}\mathbin{::}\Conid{Id}\to \Conid{Maybe}\;\Conid{Tipo}\to \Conid{Gamma}\to \Conid{Gamma}{}\<[E]%
\\
\>[B]{}\Varid{incrementaAmb}\;\Varid{n}\;\Conid{Nothing}\;{}\<[26]%
\>[26]{}[\mskip1.5mu \mskip1.5mu]\mathrel{=}[\mskip1.5mu \mskip1.5mu]{}\<[E]%
\\
\>[B]{}\Varid{incrementaAmb}\;\Varid{n}\;\Conid{Nothing}\;{}\<[26]%
\>[26]{}((\Varid{i},\Varid{e})\mathbin{:}\Varid{xs})\mathrel{=}((\Varid{i},\Varid{e})\mathbin{:}\Varid{xs}){}\<[E]%
\\
\>[B]{}\Varid{incrementaAmb}\;\Varid{n}\;(\Conid{Just}\;\Varid{v})\;[\mskip1.5mu \mskip1.5mu]\mathrel{=}[\mskip1.5mu (\Varid{n},\Varid{v})\mskip1.5mu]{}\<[E]%
\\
\>[B]{}\Varid{incrementaAmb}\;\Varid{n}\;(\Conid{Just}\;\Varid{v})\;((\Varid{i},\Varid{e})\mathbin{:}\Varid{xs}){}\<[E]%
\\
\>[B]{}\hsindent{5}{}\<[5]%
\>[5]{}\mid \Varid{n}\equiv \Varid{i}\mathrel{=}\Varid{incrementaAmb}\;\Varid{n}\;(\Varid{return}\;\Varid{v})\;[\mskip1.5mu \mskip1.5mu]{}\<[E]%
\\
\>[B]{}\hsindent{5}{}\<[5]%
\>[5]{}\mid \Varid{otherwise}\mathrel{=}\Varid{incrementaAmb}\;\Varid{n}\;(\Varid{return}\;\Varid{v})\;\Varid{xs}{}\<[E]%
\\[\blanklineskip]%
\>[B]{}\Varid{pesquisar}\mathbin{::}\Conid{Id}\to \Conid{Gamma}\to \Conid{Maybe}\;\Conid{Tipo}{}\<[E]%
\\
\>[B]{}\Varid{pesquisar}\;\Varid{v}\;[\mskip1.5mu \mskip1.5mu]\mathrel{=}\Varid{error}\;\text{\tt \char34 Variavel~nao~declarada.\char34}{}\<[E]%
\\
\>[B]{}\Varid{pesquisar}\;\Varid{v}\;((\Varid{i},\Varid{e})\mathbin{:}\Varid{xs}){}\<[E]%
\\
\>[B]{}\hsindent{5}{}\<[5]%
\>[5]{}\mid \Varid{v}\equiv \Varid{i}\mathrel{=}\Varid{return}\;\Varid{e}{}\<[E]%
\\
\>[B]{}\hsindent{5}{}\<[5]%
\>[5]{}\mid \Varid{otherwise}\mathrel{=}\Varid{pesquisar}\;\Varid{v}\;\Varid{xs}{}\<[E]%
\ColumnHook
\end{pboxed}
\)\par\noindent\endgroup\resethooks

\printbibliography


\end{document}
